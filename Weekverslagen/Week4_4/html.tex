\subsection*{Groep IO}

Scipts van app te vinden in de volgende map op de server: httpdocs/eud\\

Om de werking te zien van de app inloggen op:\\

\url{http://solarpoweredbikes.tudelft.nl/eud/app_login.php}\\

Met de volgende credentials:\\

Username: sunrise

Password: helemaalmooi\\

\noindent
Scipts van mediawebsite te vinden in de volgende map op de server: httpdocs/media\\

Om de werking van de mediawebsite te zien:\\

\url{http://solarpoweredbikes.tudelft.nl/media/index.html}\\

\subsubsection*{Deze week}

\begin{itemize}

\item
Begin aan mediawebsite.

\item
Mediawebsite een mooie menubar gegeven.

\item
Inlogscript verbeterd. Passwordcheck werkt en de volgende pagina wordt correct ingeladen.

\item
Manier van inladen van een website gaat nu via PHP ipv puur html.

\item 
Met Javascripts mouseevents aangemaakt waarmee models in webGL kunnen worden gedraaid.

\item 
Simpel model gemaakt in Blender van de oplaadpaal.

\item
Eigen bestandje gemaakt (+ parser script) waarin bepaalde parameters staan van het model waaronder Phong shading en autorotatie aan/uit. Hiermee is het gemakkelijk om elk model andere instellingen te geven.

\item
Code gereorganizeerd en in meerdere functies verdeeld voor meer overzicht en de mogelijkheid om meerdere modellen tegelijk in te laden.

\item
Aparte modelViewer script aangemaakt zodat deze universeel toepasbaar is voor zowel de app als de mediasite

\end{itemize}

\subsubsection*{Volgende week}
\begin{itemize}
\item
HTTPS is deze week nog niet naar gekeken dus dat is verplaatst naar volgende week.

\item
PHP script gegevens victron uitlezen en op de weerpagina zetten.

\item
Verder met CSS en HTML5.

\item
Cookies en sessions.

\end{itemize}
