\subsection*{Groep IO}

\subsubsection*{Deze week}
\begin{itemize}
\item Instellingen pagina aangemaakt met schuifknoppen en uitschuifmenu's
\item Statistieken pagina aangemaakt.
\item In de instellingenpagina een knop gebouwd met de mogelijkheid om uit te kunnen loggen.
\item Meerdere oplaadplekken reserveren is nu niet meer mogelijk door ingebouwde beveiliging die per gebruiker bijhoudt welk oplaadpunt door deze in gebruik is. De gebruiker wordt nu doorverwezen naar een nieuwe statuspagina waar de oplaadstatus in de toekomst gezien kan worden.
\item een paar bugs en verbeteringen aangebracht aan de CSS van de app.
\item PHP sessions werken nu naar behoren en hebben een time-out van 30 minuten.
\item modelviewer verbeterd. Kan nu gras weergeven (moet nog sterk verbeterd worden).
\item modelviewer verbeterd. Rotatie en translatie eigenschappen opnieuw ge\"implementeerd. Draaiing en translatie van modellen en bijbehorende children ging niet naar wens, werkt nu goed.
\item model van het EWI gebouw gemaakt.
\item mediawebsite flink onder handen genomen waaronder:
\item contact pagina met een map van waar het station is te vinden.
\item over ons pagina waarin het hele team moet komen te staan.
\item gewerkt aan de thesis.
\item businessplan.
\end{itemize}

\noindent De app pagina is te vinden op:\\ \url{http://solarpoweredbikes.tudelft.nl/eud/app_login}\\

\noindent Username: sunrise\\
\noindent Password: helemaalmooi\\

\noindent De media website is via de volgende link te bekijken:\\
\url{http://solarpoweredbikes.tudelft.nl/media/index.html}\\

\subsubsection*{Volgende week}
\begin{itemize}
\item HTTPS beveiliging regelen voor veilig inloggen.
\item mediawebsite afmaken.
\item statuspagina van de app invullen.
\item instellingenpagina van de app invullen.
\item weerpagina gegevens beter weergeven.

\end{itemize}
