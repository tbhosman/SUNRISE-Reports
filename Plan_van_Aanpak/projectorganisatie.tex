\section{Projectorganisatie}

Voor een goed verloop van het project is een duidelijke en overzichtelijke organisatiestructuur uiterst belangrijk. Zoals beschreven bij de Projectactiviteiten is de gehele projectgroep van 6 studenten opgesplitst in 3 subgroepen. De indeling van deze groepen is als volgt:

\subsection{Subgroepen}

\begin{itemize}
\item Groep ODROID
	\begin{itemize}
	\item Tim Hosman
	\item Job van Staveren
	\end{itemize}

\item Groep 	Server
	\begin{itemize}
	\item Nourdin Ait el Mehdi
	\item Timothy de Moor
	\end{itemize}

\item Groep I/O
	\begin{itemize}
	\item Dani\"el Brouwer
	\item David Veselka
	\end{itemize}
\end{itemize}

In elk van de groepen is er ook een eindverantwoordelijke aangesteld. Dat zijn Tim Hosman voor de ODROID en C-code, Timothy de Moor voor de Server en PHP-code en Dani\"el Brouwer voor de I/O en HTML5.\\

\subsection{Overige Functies}
Naast de eindverantwoordelijken in de subgroepen zijn er nog een aantal functies die voor een soepel verloop van het project dienen.

\begin{itemize}
	\item Eindverantwoordelijke presentaties: Eindverantwoordelijk voor het afleveren van de benodigdheden voor de presentaties. Deze functie wordt vervuld door David Veselka.
	
	\item Tijdsbewaker: De tijdsbewaker is uitgebreid op de hoogte van de planning van het project en zorgt ervoor dat deadlines bekend zijn binnen de projectgroep. Deze functie wordt vervuld door Nourdin Ait el Mehdi.
	
	\item Eindverantwoordelijke verslagen: Verantwoordelijk voor het inleveren en de beschikbaarheid van de verslagen. Deze functie wordt vervuld door Job van Staveren.
	
	\item Archivaris: Zorgt ervoor dat de GIT-repositories overzichtelijk blijven. Deze functie wordt vervuld door Job van Staveren.
\end{itemize}

\subsection{Communicatie}
Tijdens de reguliere projecturen verloopt de communicatie vooral mondeling, aangezien de projectgroep in \'e\'en ruimte aan het werk is. Voor of na deze projecturen is er ook communicatie via de Telegram app.\\
Daarnaast heeft de groep ook regelmatig contact met de opdrachtgever Peter van Duijsen. Elke maandag is er om 11.00 een bespreking gepland om de voortgang met de opdrachtgever te bespreken. Onderdeel hiervan is het korte wekelijkse verslag met de voortgang van de projectgroep(en).\\
Om communicatie met de projectgroep eenvoudiger te maken is er een Google Groups adres aangemaakt voor de projectgroep. Dit adres is: sunrise-bap@googlegroups.com. Een mail naar die adres komt dan aan bij alle projectleden.
Het archief dat bij dit project wordt opgebouwd wordt opgeslagen op de projects server van EWI (projects.ewi.tudelft.nl). Hier staan de bestanden veilig opgeslagen en zijn alleen toegankelijk voor de groepsleden. Er is voor elk van de subgroepen een repository aangemaakt om bestanden op te slaan. Daarnaast is er ook een repository voor de verslagen van de gehele projectgroep.\\