\section{Projectgrenzen}
Vanzelfsprekend zitten er grenzen aan de complexiteit van het ontwerp. Het ontwerp moet binnen de gestelde tijd gerealiseerd zijn. Om te voorkomen dat er te veel tijd in een specifiek deel wordt gestoken, wordt er steeds zo snel mogelijk naar een werkend ontwerp gewerkt. Dit houdt in dat het functionele deel een hogere prioriteit heeft dan het artistieke deel.

Het doel van dit project is het schrijven van een softwaresysteem voor een e-bike oplaadstation. De hardware die hier voor nodig is wordt ter beschikking gesteld. Dit betekent dat er niet gedaan wordt aan hardwareontwerp. Ook aan het softwaredeel worden grenzen gesteld:
\begin{itemize}
\item Website en app: deze worden gemaakt puur voor informatie over het project en om een dashboard te bieden voor de gebruiker. Beide moeten er goed uitzien maar dit mag niet ten koste gaan van de functionaliteit.
\item PHP server: deze zal de communicatie tussen de verschillende onderdelen verzorgen en gegevens verwerken. De server is beschikbaar gesteld voor het project. Het onderhouden van de server valt hier niet onder.
\item ODROID: dit onderdeel zal de beschikbare sensoren uitlezen en deze gegevens doorsturen naar de PHP server. De sensors zijn beschikbaar gesteld en hoeven binnen dit project niet gekocht en geplaatst te worden.
\end{itemize}
