\section{Samenvatting}
Aan het begin van de zomer van 2016 wordt er een oplaadpunt voor elektrische fietsen geplaatst naast het gebouw van Elektrotechniek, Wiskunde en Informatica (EWI). Dit oplaadpunt wordt echter niet volledig vanuit het net gevoed, maar ook grotendeels via zonnepanelen. Om dit oplaadpunt ook daadwerkelijk operationeel te maken is er een end-user interface nodig zodat gebruikers hun fiets op kunnen laden.\\
De projectgroep is opgesplitst in drie groepen: ODROID, Server en IO. Groep ODROID zorgt ervoor dat er met behulp van een ODROID de sensoren worden uitgelezen en stuurt de oplaadpunten aan. Groep Server zorgt ervoor dat er een administrator dashboard gemaakt wordt en maakt een server met PHP en een MySQL database. Groep IO zorgt ervoor dat er een end-user website en een app gemaakt worden.\\
Er is ingepland om voor het tussensymposium op 18 mei een werkend prototype met enkele basisfunctionaliteiten beschikbaar te hebben.