\section{Kwaliteit}

De beoordeling van dit project zal worden gedaan op basis van de kwaliteit van het eindproduct. Om een goede kwaliteit van het eindproduct te kunnen garanderen is het van belang dat de tussenresultaten, documenten en eindpresentatie ook van de juiste kwaliteit zijn. Een goede documentatie is bovendien erg belangrijk met oog op eventuele verbeteringen of aanpassingen in een later stadium van het station.

\subsection{Documenten}

Er is gekozen om alle documentatie in \LaTeX\xspace maken. \LaTeX\xspace heeft als voordeel dat de opmaak er verzorgd uitziet en verwijzingen automatisch worden veranderd indien nodig. Ook is er voor gekozen om documenten ruim voor de deadline af te hebben om zo eventuele fouten er nog uit te kunnen halen.

\subsection{Presentatie}

Tijdens het project zullen er twee presentaties worden gegeven: de midterm presentatie en de eindpresentatie met het businessplan. De kwaliteit zal hier worden gewaarborgd door op tijd te beginnen met het maken en oefenen hiervan. 

\subsection{Programmeercodes}

De kwaliteit van de programmeercodes moet exceptioneel hoog zijn. Het programma moet continu blijven draaien zonder te crashen. Om dit te kunnen bereiken zal de software uitgebreid worden getest. Verder zal de code goed becommentarieerd moeten worden zeker omdat er door meerdere personen aan wordt gewerkt.

\subsection{Eindverantwoordelijken}

Om te zorgen dat elk onderdeel van suffici\"ente kwaliteit is en dat alles op tijd af is, zijn er voor elk onderdeel eindverantwoordelijken aangewezen. Hieronder volgt een lijst met alle onderwerpen en bijbehorende eindverantwoordelijken.

\begin{itemize}

\item
Presentatie - David Veselka

\item
C-code - Tim Hosman

\item
PHP - Timothy de Moor

\item
HTML5 - Dani\"el Brouwer

\item
Documenten - Job van Staveren

\item
Deadlines - Nourdin Ait el Mehdi

\end{itemize}



