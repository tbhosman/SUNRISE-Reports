\section{Risico's}
In het project zijn een aantal risico's. Allereerst staan op de zonnepanelen en de batterij die daaraan gesloten is hoge spanningen. Dit kan een gevaar zijn voor de persoon die hiermee gaat werken, maar deze systemen zijn afgesloten, dus de risico's zijn klein.
Bovendien zou het mogelijk kunnen zijn dat de internetverbinding tussen de ODROID en de PHP-server gehacked wordt. Een goede beveiliging van de verbinding moet dat voorkomen.

Er is ook altijd nog het risico dat iemand er niet is, of te laat is. Dan moet het andere groepslid dat opvangen en loopt mogelijk vertraging op. Door het goed plannen en communiceren wordt vertraging opgevangen, ook kan er buiten normale werktijden of in het weekend worden ingehaald. Verkeerd plannen, of bijvoorbeeld een fout in de code die moeilijk op te lossen is, kan leiden tot het niet halen van de deadlines. Daarom wil de groep zo snel mogelijk een werkend prototype hebben, zodat de laatste paar weken alleen maar bestaan uit het oplossen van fouten, en het toevoegen van extra functies.
