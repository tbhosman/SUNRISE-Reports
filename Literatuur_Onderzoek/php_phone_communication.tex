\section{Communicatie tussen PHP en de smartphone}

\subsection{HTML5}

HTML5 wordt gebruikt als user interface op de smartphone of tablet waarmee een plek kan worden gereserveerd in het e-bike oplaadstation. Verder zal de interface informatie geven over bijvoorbeeld de accustatus en huidige windsnelheden. Ook zal HTML5 gebruikt worden voor de algemene site over het oplaadstation. Deze site moet dienen als een visitekaartje voor de media en ge\"interesseerden en moet dus een strak design hebben. De site zal onder andere een 3D-view van het oplaadstation bevatten. Tevens zal HTML5 gebruikt worden voor de administratieve dashboard pagina.

\begin{itemize}

\item
\cite{HTML5_manual}: Voor informatie over de beschikbare functies van HTML5.

\item
\cite{HTML5_notifications}: Om de gebruiker van bepaalde gebeurtenissen op de hoogte te stellen via de browser op de smartphone, kan gebruik gemaakt worden van notificaties.

\item
\cite{HTML5_push_api}: Om het makkelijker voor de gebruiker te maken om de notificaties niet te missen, wordt gebruik gemaakt van de Push API van HTML5.

\item
\cite{HTML5_canvas}: De website die gemaakt wordt voor de ge\"interesseerden moet er aantrekkelijk uitzien en voldoende interactie met de gebruiker hebben. Hierbij wordt gebruik gemaakt van de Canvas functie in HTML5.

\item
\cite{HTML5_main_voorbeeld_1}: Als voorbeeld voor de media website.

\item
\cite{HTML5_main_voorbeeld_2}: Voorbeeld van een site die ook werkt op breedbeeld schermen.

\item
\cite{HTML5_app_voorbeeld_1},\cite{HTML5_app_voorbeeld_2}: Applicaties die als voorbeeld kunnen dienen voor het tonen van data op een compacte en duidelijke wijze.


\end{itemize}



\subsection{PHP en MySQL}

De PHP-server wordt gebruikt om te communiceren met de Odroid en om de informatie daarvan op te slaan in een MySQL-database. Verder zal de PHP-server worden gebruikt om de bovengenoemde HTML5 pagina's te kunnen serveren aan de gebruikers.

%door middel van een website waarop algemene informatie komt over het oplaadstation en een mobiele site waarop aangemeld kan worden en een oplaadplek kan worden gereserveerd. Ook moet er een administratieve dashboard pagina komen met alle relevante informatie voor de beheerder van het oplaadstation.

\begin{itemize}
\item
\cite{php_manual}: Voor het leren van PHP zal gebruik gemaakt worden van de tutorial van PHP zelf.

\item
\cite{slidesTI1506}, \cite{database_book}: Voor de basiskennis over SQL en HTML zal gebruik gemaakt worden van de slides van het vak web- en databasetechnologie.

\item
\cite{reference_php_html5_mysql}: Naast het volgen van de tutorials zal er indien van toepassing gebruik gemaakt worden van dit boek als algemene raadpleegbron voor PHP, HTML5 en MySQL.

\end{itemize}
