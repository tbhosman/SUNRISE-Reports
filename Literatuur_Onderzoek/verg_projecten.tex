\section{Vergelijkbare Projecten en specificaties opladen}
\subsection{Inleiding}
In deze sectie staan een aantal referenties naar bronnen over specificaties van het opladen van elektrische fietsen, en over het gebruik ervan in meerdere situaties. Bovendien zijn er een aantal vergelijkbare projecten die een reserveersysteem gebruiken.

\subsection{Globale specificaties over het opladen/gebruiken van elektrische fietsen}
\begin{itemize}
\item
\cite{ftsac}: Ruwe richtlijnen van effecten zoals weer op actieradius
\item
\cite{batac}: Een opsomming van effecten die actieradius be\'invloeden
\item
\cite{feas}: Dit document is het ontwerpdocument van het laadstation waar de software voor geschreven gaat worden. Dit bevat dus een aantal specificaties waar rekening mee gehouden moet worden.
\item
\cite{chargtemp}: Batterijen zijn enkel geschikt om te laden binnen bepaalde temperaturen. Een overzicht van deze temperaturen is te vinden op de site.
\item
\cite{bss}: Een uitgebreid raport over een "Bike Sharing System" in Zweden dat onder anderen ingaat op een aantal technische specificaties van e-bikes, waaronder energieverbruik (vergeleken met o.a. auto's, pagina 63) en verbruik per kilometer (pagina 48). Deze informatie kan gebruikt worden als feedback naar de gebruikers van het station, zoals bijvoorbeeld de hoeveelheid energie dat bespaard is, of de afstand die kan worden afgelegd.
\end{itemize}
\subsection{Vergelijkbare reserveersystemen}
\begin{itemize}
\item
\cite{chargepoint}: De ChargePoint applicatie is een programma dat functies heeft die ook gebruikt kunnen worden over de te bouwen applicatie. Denk hierbij aan de real-time updates, notificaties en batterijstatus.
\item
\cite{greencharge}: Voor de GreenCharge app geldt hetzelfde als bij de ChargePoint applicatie, maar hier kan er ook nog gekeken worden naar het gebruikers interface.
\item
\cite{planyo}: Planyo is een systeem dat wordt toegepast bij reserveringen van bijvoorbeeld parkeerplaatsen. Dit is een goed startpunt voor het ontwikkelen van het systeem dat reserveringen afhandelt in het te bouwen systeem.
\end{itemize}
