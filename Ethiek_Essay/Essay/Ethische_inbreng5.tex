\subsection{Doorontwikkelen van drones}\label{chapter:doorontwikkelen}
\subtitle{Tim Hosman}
Een andere concessie tussen de twee uitersten van dit probleem is om de huidige drones eerst verder te ontwikkelen alvorens deze (weer) in gebruik te kunnen nemen. Bestaande drones mogen dan niet meer gebruikt worden totdat bewezen is dat deze voldoen aan bepaalde gestelde eisen. Er moet dan bijvoorbeeld verbeterd worden in persoonsherkenning (zoals betere sensoren of intelligentere software), nauwkeurigere wapens (bijvoorbeeld kleinere bommen, om selectiever te kunnen uitschakelen, en/of bommen die nauwkeuriger het doel kunnen raken) of de onzichtbaarheid van drones (zodat ze lager kunnen vliegen, en dus beter kunnen surveilleren en uitschakelen). Met deze ontwikkelingen kunnen vijanden effectiever worden uitgeschakeld en vallen er dus ook relatief minder burgerslachtoffers.\\

Als dit probleem wordt belicht vanuit een gevolgethisch standpunt zou deze concessie als een verbetering kunnen worden beschouwd. Het gevolg van preciezere drones is namelijk een betere uitschakeling van gevaarlijke vijanden. Dat komt niet alleen de veiligheid van het aanvallende land ten goede (die kan effectiever vijanden uitschakelen), maar in zekere zin ook het land dat aangevallen wordt. Drones worden namelijk vaak toegepast in instabiele landen waar veel onschuldige burgers niks mee te maken hebben. Nauwkeurige drones zullen deze burgers minder vaak aanvallen, en doordat de oorzaak van de onstabiliteit beter bereikt kan worden zal het land ook sneller weer stabiel kunnen worden. Het gevolg is dus meer veiligheid (en in zekere zin dus ook meer `geluk') voor de meeste inwoners uit beide landen.\\

Vanuit een plichtethisch standpunt zou je kunnen stellen dat het verantwoord is, omdat het doorontwikkelen van drones an sich niet direct zorgt voor meer doden, waardoor het bezwaar van eventueel meer doden geen rol speelt in deze afweging (het is je plicht om geen andere mensen te doden). Dit is echter wel wat kortzichtig, omdat eventuele doorontwikkeling wel degelijk invloed heeft op het aantal slachtoffers. Het kan wel effectiever zijn, maar als het om die reden veel meer wordt ingezet heeft dit alsnog als gevolg dat er meer slachtoffers vallen (in absolute aantallen misschien zelfs meer burgerslachtoffers). Het middel (het doorontwikkelen) is dus op zich verantwoord binnen de plichtethiek, het gevolg is dat mogelijk niet (als het vaker gebruikt zou worden).\\

Deugdenethiek is lastig toe te passen binnen deze oplossing, immers is de verdere ontwikkeling van drones niet direct van toepassing op de deugden zoals humaniteit. Wat deze ontwikkeling wel be\"invloedt is de deugd wijsheid. Hoe tegenstrijdig het soms kan klinken, er zijn veel ontdekkingen te danken aan tijden van oorlog. Denk bijvoorbeeld aan de Turing machine of penicilline. Ontwikkeling van drones kan op diezelfde manier veel kennis vergaren, sneller dan in tijden zonder oorlog. Het zal verre van de doorslaggevende reden zijn, maar het valt op z'n minst te noemen.\\

Alles welbeschouwend valt er veel goeds te zeggen over het doorontwikkelen van drones voordat deze toegepast mogen worden in oorlog. Het zou in theorie effectiever moeten zijn dan huidige drones en er zouden relatief minder burgerslachtoffers moeten vallen. Er zitten nog wel haken en ogen aan: wat voor eisen moeten er gesteld worden aan deze toekomstige drone? Hoe gaat dit getest worden? Hoe lang gaat die ontwikkeling duren? Bovendien kan het zijn dat drones z\'o effectief zijn dat ze op zeer grote schaal worden toegepast. Het gevolg kan zijn dat er in absolute aantallen alsnog erg veel burgerslachtoffers vallen.
