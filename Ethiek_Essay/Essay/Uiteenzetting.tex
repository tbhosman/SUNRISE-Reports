\section{Uiteenzetting}\subtitle{Timothy de Moor}
Bij deze probleemstelling zijn een aantal partijen betrokken. Deze hebben elk hun eigen belangen bij deze casus. Allereerst zijn de burgers betrokken. Zij willen leven in veiligheid, en het liefst zonder dat zij zich daarbij de hele tijd bang hoeven te voelen. Ook zijn de overheden, die de drones bezitten en ontwikkelen, betrokken bij deze probleemstelling. Zij willen de veiligheid van de inwoners van hun landen kunnen garanderen. Daarnaast willen zij een goed imago hebben en de kosten van oorlogsvoering kunnen beperken. De effectiviteit van de aanvallen is ook \'e\'en van hun belangen.\\

De volgende waarden zijn in het geding: veiligheid, macht, geld, etc.. De veiligheid voor de eigen burgers en ook zoveel mogelijk voor de burgers van het andere land staan voorop. Daarnaast willen 
leiders zoveel mogelijk macht krijgen en het liefste ook nog voor zo laag mogelijke kosten.\\

Er zijn een aantal feiten die mee kunnen worden genomen in een ethisch oordeel over drones. Allereerst is het vliegen van drones over langere tijd gezien goedkoper \cite{pdf_ethiek_drone_aanvallen}, hoewel ze meer mensen nodig hebben om in de lucht te blijven. Daarnaast kunnen straaljagerpiloten niet de hele dag door vliegen; de straaljagers moeten dus vaker terug naar de uitvalsbasis. Drones daarentegen kunnen veel langer in de lucht blijven als hun piloten elkaar afwisselen\cite{briefing}. Daardoor hoeven ze juist minder vaak terug te keren naar de basis. Ook zijn de straaljagerpiloten constant in gevaar als ze in de lucht zijn. Wanneer er zich problemen voordoen met hun toestel zijn de overlevingskansen soms niet altijd even groot, en zeker niet in het geval dat de piloot zich boven vijandelijk gebied bevindt. Als de piloot zijn schietstoel gebruikt als laatste redmiddel, loopt hij het aanzienlijke risico om gevangengenomen te worden. Bij het gebruik van drones is dit geen risico meer en worden er geen militairen in gevaar gebracht.\\

Ook kan de communicatie tussen drones en mensen misschien problemen geven. Mensen zijn al gauw geneigd om machines minder te vertrouwen dan andere mensen \cite{briefing}. Bovendien kan de verbinding tussen de drone en de controlekamer problemen geven, waardoor de drone onbestuurbaar wordt. Of de camera valt uit en de piloot heeft geen idee waar de drone is. Deze punten kunnen allemaal worden meegenomen in uiteindelijke conclusie om zo een antwoord te kunnen vinden op de probleemstelling.
