\section{Ethische Inbreng en Overwegingen}
\subsection{Zwart-Wit Strategie}\label{chapter:zwart-wit}
\subtitle{Job Van Staveren}
Overduidelijk de meest radicale oplossingen vloeien voort uit het kader van de zwart-witstrategie. In dit geval gaat dat maar deels op omdat \'e\'en van de uitersten de huidige situatie is. Intu\"itief zijn de uiterste oplossingen vaak niet de beste. Dat geldt ook voor deze probleemstelling. Het stoppen van het gebruik van militaire drones zal niet eenvoudig zijn. De ontwikkeling stoppen is bijna onmogelijk. Aan de andere kant is er de huidige situatie die we kunnen laten zoals het is. Of d\'at wenselijk is is ook nog maar de vraag. Het internet staat vol met voorbeelden van drone-aanvallen waarbij vele burgerslachtoffers vallen. Neem bijvoorbeeld het artikel uit The Guardian van 24 november 2014 \cite{guardian}. Dit artikel laat een aantal cijfers zien van aanvallen die gedaan zijn waarbij is gebleken dat de aanvallen heel wat minder precies zijn dan dat men in eerste instantie zou denken. Totaal zijn daar 41 mensen aangevallen, maar er kwamen 1147 mensen om. Dit artikel is vermoedelijk wel enigszins gekleurd in de zin dat het waarschijnlijk selectief is in het laten zien van de cijfers. Desalniettemin laat dit wel zien dat het doorgaan op deze manier ook niet gewenst is.\\

%Ethische inbreng
%Gevolgethiek/utilisme
Vanuit het utilisme lijkt er geen eenduidig antwoord te zijn op de probleemstelling. De hoofdreden daarvoor is dat het lastig is om waarden zoals veiligheid te kwantificeren. Ook is er voor beide uitersten wel wat te zeggen. Zo heeft het blijven gebruiken, en dus ook verder ontwikkelen van drones, natuurlijk het voordeel dat de aanvallen met een grotere precisie uitgevoerd kunnen worden. Zo kunnen gevaarlijke kopstukken van terroristische groeperingen onschadelijk gemaakt worden. Dat heeft als gevolg dat vooral mensen die in landen met een terroristische dreiging wonen een zekere mate van veiligheidsgevoel hebben. Of dat echter opweegt tegen het aantal doden dat door aanvallen wordt veroorzaakt valt te betwijfelen.\\

%Plichtethiek
De plichtethiek lijkt deze zorg te onderstrepen. Het is duidelijk dat het niet de plicht van de mens is om te doden. Aangezien er in de hedendaagse oorlogen nog steeds vele, al dan niet onschuldige, slachtoffers vallen, is oorlog voeren dus vanuit de plichtethische optiek niet toelaatbaar. Drones die gebruikt worden tijdens een oorlog zijn mede de veroorzakers van het vallen van slachtoffers. Het gebruik en de ontwikkeling van drones voor militaire doeleinden is daarom niet ethisch verantwoord. \\

%Deugdenethiek
Ook het denkkader van de deugdenethiek geeft geen uitsluitsel. Aan de ene kant is er te zeggen dat het tot een taak van de overheid behoort om ervoor te zorgen dat de burgers en militairen in veiligheid kunnen leven/werken en dus in wezen een vorm van bescherming dienen te bieden. Het bieden van bescherming valt ook onder de noemer van de deugden. Gebruik van drones en ontwikkeling zal, met het oog op bieden van bescherming van militairen en burgers in het eigen land, zo dus ook gerechtvaardigd zijn. Wanneer echter de deugd van de bescherming wordt beschouwd met het oog op de burgers in conflictgebieden, is het voorbeeld uit \cite{guardian} juist weer een indicatie dat het gebruiken en ontwikkelen van drones toch minder ethisch verantwoord is dan het in eerste instantie lijkt.\\

%Verantwoordelijkheidsethiek
Wanneer er echter gekeken wordt naar de verantwoordelijkheidsethiek treedt er een probleem op: verantwoordelijkheidsethiek gaat expliciet over 'de ander' in de vorm van een individu. Maar in het geval van de probleemstelling gaat het eigenlijk alleen over twee of meer landen die zich in een conflictsituatie bevinden. Wat dat betreft lijkt er hier dus geen antwoord te zijn. Het is natuurlijk nog wel mogelijk om een individu te defini\"eren in een zekere zin. De individu zou zo als een 'onschuldige burger' gedefini\"eerd kunnen worden. Aangezien dat in dit geval niet van heel veel toegevoegde waarde is ten opzichte van het bovenstaande, wordt de verantwoordelijkheidsethiek hier buiten beschouwing gelaten.\\


%Voor deze burger zou het zeker van belang zijn dat de drones verder ontwikkeld worden. Een betere precisie zorgt voor minder onschuldige slachtoffers. Het risico is echter wel aanwezig dat het gebruik van drones de drempel verlaagt tot aanvallen en er dus meer aanvallen uitgevoerd gaan worden. Opnieuw zijn er dus handelwijzen die lastig met elkaar te verenigen zijn.\\


Zoals uit de analyse hierboven is gebleken zijn zeker niet alle denkkaders zinvol in het geval van de uitersten en is er zeker geen eenduidig antwoord. De verantwoordelijkheidsethiek is niet relevant in dit geval. Ook de plichtethiek geeft een niet al te bruikbaar antwoord. Oorlog is nu eenmaal een feit en hoewel het natuurlijk gewenst is daar direct een einde aan te maken, is het allesbehalve realistisch om dat ook daadwerkelijk te doen en te kunnen. De gevolgethiek en deugdenethiek bieden in dit vraagstuk betere argumenten, zij het dat de methoden zelf niet eenduidig zijn.\\

Wanneer de haalbaarheid van beide uitersten wordt beschouwd is het ``laten zoals het nu is'' uiteraard eenvoudig. Het stoppen met de ontwikkeling en gebruik van drones is echter niet een realistisch denkbeeld. Daar komt ook nog bij dat de uitersten beiden niet volledig ethisch te verantwoorden zijn. Er zal dus meer een oplossing gezocht moeten worden in de vorm van een tussenweg.
