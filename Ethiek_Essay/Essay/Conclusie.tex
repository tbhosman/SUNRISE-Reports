\section{Conclusie}
\subtitle{Tim Hosman \& Dani\"el Brouwer}
Kijkend naar de uitersten (sectie \ref{chapter:zwart-wit}) kan worden gesteld dat de huidige situatie ethisch niet te verantwoorden valt. De huidige situatie volhouden is twijfelachtig, omdat er vermoed wordt dat er veel onschuldige burgers bij omkomen. Hoewel helemaal verbieden theoretisch gezien een optie is, is het in de praktijk moeilijk haalbaar om daar mondiaal een wet van te maken, en als \'e\'en land weigert om mee te werken is het onwaarschijnlijk dat andere landen dat wel doen (zij worden dan immers benadeeld).\\

Een concessie lijkt daarmee dus ethisch het beste te onderbouwen. Een limiet op het aantal slachtoffers is theoretisch een ethisch verantwoorde optie (sectie \ref{chapter:limiet}). Het is alleen de vraag hoe haalbaar dit is. Hoe gaat dit bijgehouden worden? Wat is een burger en wat is een vijand? Bovendien kan een land sjoemelen met cijfers en onbedoeld aangevallen burgers bestempelen als vijand. Aan de andere kant kan je stellen dat deze regel er hoogstens toe kan leiden dat de situatie hetzelfde blijft en anders \'iets verbetert. Meer burgerslachtoffers zijn met het invoeren van deze regel zeer onwaarschijnlijk. Het kan wel voorkomen dat de limiet wordt bereikt en daarmee niet meer aangevallen mag worden in kritieke situaties.\\

Een betere opleiding voor dronepiloten is ook een optie (sectie \ref{chapter:beter_opleiden}), het zorgt mogelijk voor een betere bewustwording over de ernst van de daad. Of dit in de realiteit een ree\"el verschil maakt in het aantal burgerslachtoffers valt te betwijfelen, maar ook voor deze oplossing geldt dat het onwaarschijnlijk is dat het de situatie verslechterd. De enige reden om niet beter op te leiden is het financi\"ele aspect.\\

Een drone die alleen kan surveilleren zet een drempel tussen het waarnemen van een potenti\"eel gevaar en het uitvoeren van een aanvallende actie (sectie \ref{chapter:surveillance}). Daarmee zal er alleen aangevallen worden op belangrijke kopstukken of groepen met weinig onschuldige burgers. Er is nu wel een risico dat de situatie verandert tussen het observeren en aanvallen. Bovendien is het dan de vraag of drones nog effectief genoeg zijn om in gebruik te nemen.\\

Het doorontwikkelen van drones (sectie \ref{chapter:doorontwikkelen}) levert ethisch gezien een betere situatie op dan de huidige. Doorontwikkelen zorgt namelijk voor drones waarmee beter een situatie ingeschat kan worden en waarmee nauwkeuriger aangevallen kan worden. Dat is op lange termijn een win-winsituatie voor zowel het aanvallende als het aangevallen land. Het cre\"eert wel een situatie op korte termijn waarin drones niet langer toegepast kunnen worden, ze moeten immers eerst worden verbeterd. Bovendien is het discutabel wanneer een drone als ``goed genoeg" wordt beschouwd.\\

De laatste stap in de ethische cyclus is het formuleren van de moreel verantwoorde handeling als antwoord op de probleemstelling. \\
Om het gebruik en de ontwikkeling van drones voor militaire toepassingen ethisch te verantwoorden moet er tenminste ge\"investeerd worden in de opleiding. Dit heeft ethisch gezien geen nadelen. Eenzelfde redenatie geldt voor het doorontwikkelen van drones voor betere precisie; het tijdelijk stoppen met gebruiken van drones zou eventueel wel tijdelijk invloed op de veiligheid kunnen hebben.
Wanneer deze twee handelingsopties toegepast worden is het gebruik en de ontwikkeling van militaire drones ethisch te verantwoorden. De overige twee opties, de limiet op burgerslachtoffers en het gebruiken van drones voor surveillance, zouden zeker van toegevoegde waarde kunnen zijn. Het grootste probleem bij deze oplossingen is echter wel de haalbaarheid en de eventuele (tijdelijke) gevolgen met betrekking tot de veiligheid. Desalniettemin hebben alle vier de mogelijkheden het grote voordeel dat deze eenvoudig en zonder grote consequenties teruggedraaid kunnen worden.\\