\subsection{Limiet op het aantal burgerslachtoffers}\label{chapter:limiet}
\subtitle{Nourdin Ait El Mehdi}
Een limiet op het aantal burgerslachtoffers kan gezien worden als een oplossing, maar hoe bepaal je hoeveel mensenlevens er verloren mogen gaan? Wordt er een harde grens opgesteld in gehele getallen of wordt de grens bepaald aan de hand van een percentage van de hoeveelheid inwoners in een land? \\

Het klinkt heel moreel onaanvaardbaar om een getal te plakken aan het aantal mensenlevens dat verloren mag gaan. Is het erg als deze mensen dood gaan? Worden deze mensen als minder gezien? Door een limiet te zetten op het aantal burgerslachtoffers wordt ervoor gezorgd dat er minder doden vallen. Er moet immers gestopt worden met de drone aanvallen op het moment dat het limiet bereikt is. Op basis van de plichtethiek kun je hier een morele wet van maken, maar gaat deze wet niet in tegen andere wetten zoals gelijkheid? Een limiet op burgerslachtoffers kan betekenen dat deze mensen als minderwaardig worden gezien, je drukt het immers uit in een getal. Aan de andere kant zou je dit getal meer als streefgetal kunnen zien, net zoals dat wordt gedaan met bijvoorbeeld verkeersslachtoffers.\\


Vanuit het oogpunt van het grootste geluk voor het grootste aantal mensen kun je de mensen opdelen in twee groepen: de bevolking van het aanvallende land en de bevolking van het land dat word aangevallen. In het geval dat er zowel een relatief aantal burgerslachtoffers mag vallen (dus ten opzichte van het aantal vijanden) als absoluut, dan is deze oplossing vanuit een gevolg- en verantwoordelijketisch standpunt verantwoord. Immers, er wordt relatief weinig geluk 'ontnomen' in het aangevallen land (burgerslachtoffers) en er komt relatief veel geluk voor terug in beide landen (veel vijandelijke doden). Hiervoor zouden drones wel nauwkeuriger moeten worden, want op dit moment zijn drones nog niet ver genoeg ontwikkeld om deze aanvallen accuraat uit te voeren \cite{papers}. Het gevolg is dat de onschuldige burgers in het land dat je aanvalt gevaar lopen, wat een pijn voor de burgers van dat land tot gevolg heeft.\\

Het is een moeilijke afweging die tegen de deugden van de mens in gaat. Want kies je nu voor jouw burgers ten koste van andere burgers? Kijkend naar de deugdenethiek zou doden in het algemeen niet geheel goedgekeurd worden, maar minder onschuldige doden is wel een stap in richting van meer humaniteit. Bovendien zorgt een limiet op het aantal burgerslachtoffers voor meer veiligheid in het land dat wordt aangevallen. In het land dat aanvalt zou de veiligheid kunnen afnemen, omdat er bij twijfel vaker ingehouden zal worden en er dus meer potenti\"ele tegenstanders blijven leven. Dit kan weer leiden tot het langer instabiel blijven van het land dat wordt aangevallen, waardoor de veiligheid van onschuldige burgers alsnog in het geding komt.\\
  
Door het doorontwikkelen van drones kan deze regel buiten spel gezet worden. Als drones nauwkeuriger zijn zullen ze ook minder slachtoffers maken, dan is er geen relatieve limiet nodig. Naast dat de drones zelf niet accuraat genoeg zijn, ligt er ook zeker een verantwoordelijkheid bij het leger die te snel zonder genoeg informatie de drones in zet om een aanval te doen. Het gemak waarmee de drones gebruikt kunnen worden zorgt ervoor dat er meer risico genomen word. Door het aantal burgerslachtoffers te beperken in absolute zin kan overgebruik worden voorkomen. Een limiet in relatieve en absolute aantallen burgerslachtoffers zou dus een potenti\"ele oplossing zijn.