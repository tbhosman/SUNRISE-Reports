\subsection{Drones alleen gebruiken voor surveillance}\label{chapter:surveillance}
\subtitle{Dani\"el Brouwer}
Een andere mogelijke oplossing voor het probleem is om internationale regels op te stellen, waar bijvoorbeeld het verbieden van gewapende drones onder valt. Hierdoor kan deze alleen nog maar voor surveillance doeleinden worden gebruikt. Men kan stellen dat dit bevorderend is voor de veiligheid omdat een directe aanval met drones niet meer (gemakkelijk) mogelijk is. Dit geeft de aanvallende partij de mogelijkheid om de aanval `beter' voor te bereiden en wellicht met grondtroepen de dader te arresteren in plaats van te doden. Tevens zal er uitvoeriger over de aanval nagedacht moeten worden omdat er nu eigen levens (de soldaten) op het spel staan. Ondanks dit zal de beslissing om een aanval uit te voeren nog steeds op basis van gegevens die de drone verzamelt zijn en kun je dus stellen dat hiermee het probleem eigenlijk niet wordt opgelost. Verder is het maar de vraag in hoeverre het haalbaar is om een internationale regel op te stellen die gewapende drones verbied; laat staan het controleren op het naleven van deze regel.\\ 

De handelingsmethode om gewapende drones te verbieden kan uit verschillende ethische gezichtspunten worden getoetst. De gevolgenethiek bijvoorbeeld waar gekeken wordt naar het grootste geluk voor het grootst aantal mensen. In dit geval is niet duidelijk te defini\"eren wie hierbij het meeste geluk zal ervaren, het is namelijk niet te zeggen of de handelingsmethode \"uberhaupt er voor zal gaan zorgen dat de veiligheid omhoog gaat, laat staan wie er dan het meeste van gaat profiteren. Indien het verbieden van gewapende drones, zoals hierboven genoemd, er inderdaad voor zou zorgen dat aanvallen `preciezer' en beter voorbereid worden uitgevoerd zou dit positieve gevolgen hebben voor zowel de burgers van het aanvallende land als de burgers in het land waar de misdadigers zich bevinden. Dan zou de handelingsmethode dus voor het grootste geluk zorgen voor een zo groot mogelijk aantal mensen en dus ethisch verantwoord zijn.\\

Als men het vanuit de plichtethiek bekijkt is het middel, drones voor surveillance doeleinden, niet verkeerd. Tenslotte zijn de surveillance drones niet in staat mensen te doden, thans niet op een directe manier en wordt er aan de plicht om niet te doden voldaan. Tevens is de regel om drones voor surveillance doeleinden gebruiken in de context van oorlogsvoering (dus niet kijkend naar privacy) goed universaliseerbaar. Het zal de veiligheid bevorderen, en het veilig houden van de burgers in een land kan als een algemene plicht worden gezien van het leidend orgaan van een land.\\

Wanneer vanuit de deugdenethiek wordt gekeken zullen klassieke militaire deugden zoals moed en eergevoel een rol spelen. Het doden van mensen achter een schermpje via een drone zal minder moed vergen dan de daad van dichtbij te ervaren. Als drones niet bewapend zijn, zullen er dus moedige militairen moeten zijn die de aanval ter plaatsen uitvoeren. Bovendien zal het trotseren van het gevaar voor de veiligheid van eigen burgers, een hoger eergevoel geven.\\

Als het verbieden van gewapende drones inderdaad zal leiden tot preciezere en beter voorbereide aanvallen zal volgens zowel de gevolgen- als verantwoordelijkheidsethiek het een goede oplossing zijn. Maar dit hoeft niet het geval te zijn. Indirect zou de veiligheid niet hoeven te verbeteren. De informatie verzameld door de drone bepaald namelijk nog steeds of er een aanval zal worden uitgevoerd en een gevechtsvliegtuig zou ook een bom kunnen gooien, direct nadat de drone heeft ``gesurveilleerd". Het feit is er wel dat een aanval moeilijker gemaakt wordt dan wanneer de drone zelf aan kon vallen. Wanneer gekeken wordt vanuit de deugdenethiek zal moed en eergevoel van soldaten zeker wel een rol spelen. Echter hebben deze deugden geen directe correlatie met de verhoging van de veiligheid van de burgers van het aanvallende en aan te vallen land. Deze redenering is dus niet relevant in deze situatie. Via de beredenering van de plichtethiek zal de handelingsmaatregel niet slecht zijn omdat surveillance drones niet kunnen doden. Hierbij is wel op aan te merken dat er indirect door de surveillance alsnog wordt gedood, alleen dan met alternatieve wapens, en zal het dus niet direct tot meer veiligheid zal leiden.\\

Het verbieden van bewapende drones waardoor deze alleen nog maar voor surveillance doeleinden kan worden gebruikt zal niet voor een onveiligere situatie zorgen dan bewapende drones, en kan zelfs voor veiligere situaties zorgen indien er hierdoor met meer voorzichtigheid wordt gehandeld door het aanvallende land. Het laatste is discutabel, maar het feit dat de situatie er in theorie er niet onveiliger op zal kunnen worden, maakt deze handelingsmogelijkheid wel redelijk en ethisch verantwoord. 
