\section{Handelingsopties}
\subtitle{Job van Staveren}
De volgende stap in het komen tot een standpunt wat betreft de morele vraagstelling is het opstellen van een aantal handelingsmogelijkheden. Deze mogelijkheden zijn gebaseerd op een viertal strategie\"en: de zwart-witstrategie, aankaarten op hoger niveau, strategie van samenwerking en de strategie van het klokkenluiden.
In het geval van de probleemstelling zijn hier twee strategie\"en van minder grote waarde: het aankaarten op hoger niveau en het klokkenluiden. De eerste vooral omdat het hier ontwikkeling betreft die voornamelijk in gang gezet wordt door overheden. Zij willen dat hun legermacht beschikking heeft over drones. De tweede vooral omdat de vraagstelling niet zo zeer gaat over bepaalde misstanden.\\

De eerder genoemde zwart-wit strategie levert het denkkader waarin er slechts uitersten mogelijk zijn. Toegepast op het vraagstuk komt dat hier concreet neer op het stoppen met gebruik van drones voor militaire doeleinden. Dat kan ook nog extremer; zelfs het stoppen met de ontwikkeling van drones wordt in dit kader als een oplossing gezien. Uiteraard is er ook het andere uiterste; dat is het doorgaan met de huidige situatie dat dus eigenlijk neerkomt op helemaal geen actie ondernemen. Dat het `laten zoals het is' eigenlijk een uiterste handelwijze is, laat nog weer eens zien dat het dringend noodzakelijk is om een degelijk ethisch oordeel over deze probleemstelling te hebben.\\

Als tweede denkkader voor het vinden van oplossingen wordt de strategie van samenwerken in beschouwing genomen. Hierbij wordt met de betrokkenen bij de ethische vraagstelling gezocht naar een oplossing zodat iedereen daar baat bij heeft. Eigenlijk moet er dan gezocht worden naar een `gulden middenweg'.\\ 

Een eerste handelwijze zou bijvoorbeeld het instellen van een limiet op burgerslachtoffers kunnen zijn bij het gebruik van drones. Hetzij via een absoluut getal, een percentage of een combinatie daarvan. Zo is het voor de aanvallende partij nog steeds mogelijk om aanvallen uit te voeren maar daarmee is de hoeveelheid gebonden aan een maximum aantal burgerslachtoffers.\\

Als tweede handelingsoptie kan het beter opleiden van dronepiloten genoemd worden. Aangezien drones vanaf een beeldscherm bestuurd kunnen worden, is het natuurlijk niet vreemd dat dronepiloten minder opleiding genieten dan de straaljagerpiloten. Op zich klinkt dat natuurlijk niet zo vreemd, maar de impact van de aanvallen van straaljagers en drones is vergelijkbaar. Het wordt zelfs vergeleken met het besturen van een commercieel vliegtuig door een buschauffeur \cite{ethical_analysis}.\\

Een derde optie is een bepaalde mate van regulering op het gebruik van drones aanbrengen door middel van internationale wetten. Net zoals dat bijvoorbeeld voor clusterbommen en gifgas is gebeurd. Daarbij hoeft het niet om een expliciet verbod te gaan, maar is waarschijnlijk het opleggen van bepaalde restricties effectiever. Denk hierbij aan bijvoorbeeld het verbieden van het uitrusten van drones met wapentuig. Dan kunnen drones slechts gebruikt worden voor surveillance doeleinden.\\

Ten slotte is het doorontwikkelen/optimaliseren van drones nog een optie. Het gaat dan specifiek om het doorontwikkelen van de bestaande functies. Uitbreiden van de functionaliteit dus uitdrukkelijk niet. Doorontwikkelen kan er zo voor zorgen dat aanvallen preciezer uitgevoerd kunnen worden en de zogenaamde `collateral damage' beperkt kan worden.\\

