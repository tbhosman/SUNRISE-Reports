\section{Inleiding}\subtitle{Timothy de Moor}
In het leger wordt altijd geprobeerd om de tegenstander voor te blijven, bijvoorbeeld door middel van vernieuwende techniek. E\'en van de nieuwe concepten daarbij is het gebruik van drones. Zij kunnen autonoom vliegen en kunnen door middel van een afstandsbediening bestuurd worden. Deze drones vervangen daarbij deels de straaljagers, die door piloten worden gevlogen.\\
Deze drones hebben op dit moment een groot nadeel: Ze zijn nog niet nauwkeurig genoeg, dus er worden teveel burgerslachtoffers gemaakt. Daarom zal dit essay zich buigen over de volgende morele vraag:

{\it{Onder welke voorwaarden is het gebruik en de ontwikkeling van drones voor militaire toepassingen moreel aanvaardbaar, zelfs als dit nadelige effecten kan hebben voor de gewone burger?}\vspace{5mm}}

Het ethische probleem is hier de vraag of drones, en het verder gebruik hiervan, moreel goedgekeurd kan worden of dat het leger nog steeds gebruik moet maken van de conventionele technieken waaronder het inzetten van straaljagers.
Op dit moment is het vrij duidelijk voor wie deze ethische vraag een probleem is, namelijk voor landen die aanvallen met drones en landen die worden aangevallen door drones. In de toekomst echter, is het niet goed aan te wijzen voor wie dit een probleem is. De landen waar dit van toepassing is veranderen namelijk door de jaren heen, dus het antwoord op deze vraag zou zoiets als ``iedereen" zijn.
De morele aard van deze vraag is voornamelijk de veiligheid. Dit is de grootste waarde die in het geding is. Binnen deze veiligheid moet worden afgewogen tegen het alternatief van drones: straaljagers. Er zijn nog andere waardes 'in het geding, hier wordt echter niet op ingegaan. 
