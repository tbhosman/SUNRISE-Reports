\subsection{Strengere opleiding voor dronepiloten}
\subtitle{David Veselka}
\label{chapter:beter_opleiden}
Om de betrouwbaarheid van de drone-aanvallen te bevorderen, kunnen er strengere eisen worden gesteld aan de opleiding van de dronepiloten. Zo moet er duidelijk gemaakt worden dat de piloten nog steeds in direct contact met de buitenwereld staan. Het gevaar schuilt namelijk dat het vernietigen van doelen, wat gepaard kan gaan met burgerslachtoffers, slechts met een druk op de knop gebeurt zonder dat het ethische aspect in acht wordt genomen. Het gevolg hiervan is dat er makkelijker doelen vernietigd worden zonder er eerst een goede beslissing over te vormen. Hierbij moet gekeken worden naar het aantal burgerslachtoffers dat mogelijk gemaakt wordt. Als piloten er niet goed genoeg in getraind worden om zulke ethische beslissingen te nemen, hoewel preciezer, alsnog zorgen voor meer onnodige slachtoffers. De piloten kijken de dood nooit direct in de ogen en hoeven dus mentaal minder weerbaar te zijn. Omdat ze nooit in de frontlinie te vinden zijn, zijn de fysieke eisen die gesteld worden automatisch lager.\\

De manier waarop doelen worden vernietigd met drones verschilt niet veel van de reguliere manier: het doel wordt opgespoord met geavanceerde apparatuur en wordt vernietigd met een druk op de knop. Waar wel een wezenlijk verschil zit, is in de afstand tussen slagveld en bestuurder. De dronepiloot zou in de Verenigde Staten kunnen zitten en een drone kunnen besturen die in het Midden-Oosten vliegt. De camerabeelden worden op beeldschermen weergegeven en de drone met joystick of gamecontroller bestuurd. Hierdoor gaat het lijken of het een spelletje is en wordt het ethische aspect makkelijk vergeten. De gevolgen van het doden van mensen zijn er namelijk niet direct, net zoals in games. Een meer kritische kijk op de opleiding van dronepiloten kan ervoor zorgen dat de mensen in opleiding zich meer bewust worden van de gevolgen. Hierbij moet afstand genomen worden van de ``game sfeer'' en de piloten in spe een goede ethische training worden gegeven.\\

In het licht van de plichtethiek kan gekeken worden naar de regels en wetten die er komen kijken bij het strenger opleiden van dronepiloten. Ook kan het universaliseringsbeginsel van Kant worden geillustreerd aan de hand van de genoemde handelingsmogelijkheid. De maxime hierbij luidt: `Om het gebruik van drones voor militaire doeleinden te verantwoorden worden strengere eisen gesteld aan de dronepilotenopleiding'. Door het opleiden van de piloten worden drones breder inzetbaar en kan de trend naar meer automatisering in gang worden gezet. De opleiding zal met strengere eisen ervoor zorgen dat piloten zich bewuster van zijn dat doelen raken met drones nagenoeg hetzelfde effect heeft als een piloot die dit vanuit een normale straaljager doet. De opleiding zelf is niet slecht, dus universaliseerbaar.\\

De gevolgenethiek beschrijft dat de middelen ondergeschikt zijn aan het doel. Met het strenger maken van de opleiding voor dronepiloot wordt als doel bereikt dat de piloten zich meer bewust zijn van de gevolgen van hun acties. Er wordt beter omgegaan met het feit dat mensen doden vanuit een drone op afstand net zo goed doden is als op de reguliere militaire manier. Dit is positief omdat er meer onschuldige burgerlevens kunnen worden gespaard. Er wordt meer meegeleefd met de mensen die niet bij de oorlog betrokken willen zijn in plaats van ze als spelobjecten te zien. Deze kunnen allemaal gezien worden als goede gevolgen. De slechte gevolgen zijn dat het meer geld kost om toekomstige piloten te selecteren en op te leiden. Hierbij zijn echter geen mensenlevens in het spel. Als de goede en slechte gevolgen tegen elkaar worden afgewogen komt het neer op mensenlevens of geld. Hierbij zijn mensenlevens een stuk belangrijker dan het geld dat er nodig is om die mensenlevens te bewaren.\\

Door het aanscherpen van de opleiding en het cre\"eren van bewustwording over het doden van mensen met drones kan de bestuurders worden aangeleerd hoe de gulden middenweg te kiezen is. Het is de bedoeling dat ze effectief de opdracht kunnen uitvoeren, zonder hierbij onnodig mensenlevens in gevaar te brengen. De deugd van het respecteren van andermans leven kan met het verbeteren van de opleiding worden aangeleerd. Met het nastreven van deze deugd wordt het behoud van onschuldige burgerlevens nagestreefd.\\

Om verder in te gaan op de burgers, deze kunnen bij de verantwoordelijkheidsethiek gezien worden als ``de ander". Deze mag niet gebruikt worden voor eigen doeleinden en is nooit ongewild te integreren voor eigen doeleinden. Hieronder valt ook het beschikken over de mensenlevens door onzorgvuldig gebruik van drones wat voortkomt uit minder goed opgeleide dronepiloten. Bij een aanval hebben zij geen kans om terug te slaan of om op een andere manier vergelding te krijgen. Bij het negeren van deze levens wordt geen rekening gehouden met de ander, wat tegen de verantwoordelijkheidsethiek in gaat. Als er strengere eisen worden gesteld aan de opleiding van dronepiloten kan dit de bewustwording van de gevolgen van het doden van mensen vergroten. Dit is noodzakelijk, want anders worden er mensenlevens op het spel gezet. Als de piloten het besturen van de drone gaan zien als game, vergeten zij de waarde van een mensenleven en houden zij geen rekening met de ander. Dit is een kwalijke zaak, omdat hiermee het genadeloos afslachten van onschuldige mensenmassa's wordt vergemakkelijkt.\\

Aan de andere kant kan worden gekeken naar het financi\"ele gedeelte. De overheden hebben meer baat bij goedkope oorlogsvoering dan bij een paar slachtoffers meer of minder. Maar is dit wel moreel goed om zo te handelen? Is een onschuldig mensenleven niet veel belangrijker dan geld? Drones en de mensen die ze besturen moeten ingezet worden voor het doel waarvoor ze bedoeld zijn, namelijk vijandelijke doelen neerhalen. Hierbij moet voorop staan dat er zo weinig mogelijk slachtoffers vallen. Geld of macht mag nooit de overhand over het leven van anderen hebben.\\ \\
