\section{Structuur}
\subsection{De onderzoeksvraag}
Moeten we doorgaan met het ontwikkelen van drones voor militaire toepassingen?
Dit weegt de veiligheid van de aanvallers af tegen de veiligheid van de verdedigers (dit wordt uitgewerkt in de inleiding).

\subsection{Voorargumenten}

\begin{itemize}

\item
Minder slachtoffers aan de ``drone" kant; er zijn geen piloten aanwezig die kunnen sneuvelen.

\item
Drones kunnen langer effectief in de lucht blijven. Drone piloten kunnen wisselen van wacht, straaljagerpiloten kunnen dat niet.

\item
Bij verantwoordelijk gebruik vallen er minder burgerslachtoffers.

\item
Meer belangrijke tegenstanders kunnen worden gedood omdat drones preciezer kunnen werken.

\item
Piloten van drones kunnen juist meer last krijgen van post traumatische stress stoornissen. De camera blijft op de plek gericht waar de ontploffing plaatsvond.

\item
Een neergestorte drone is van geen waarde voor het aangevallen land (mits goed beveiligd, zoals zelfvernietiging). Een neergestorte straaljagerpiloot kan gebruikt worden als gegijzelde.

\end{itemize}

\subsection{Tegenargumenten}
\begin{itemize}
\item
De stelling dat drones burgerslachtoffers verlagen is misleidend. Natuurlijk zal de preciezere technologie in verhouding minder burgers doden maar het is nu gemakkelijker om een aanval uit te voeren want er staan minder eigen levens op het spel. Iets wat eerst niet tot een doel behoorde zal door gemak nu wel tot een doel behoren.

%\item
%Drone warfare heeft ook implicaties. Burgers in landen die worden aangevallen zullen met meer angst leven. Iedere dag het idee dat een geautomatiseerde (door mensen geprogrammeerde) drone `zomaar' kan aanvallen is niet prettig.

\item
``Echte" piloten hebben een veel betere opleiding. Hierdoor is bij bestuurders van drones de kans op fouten groter.

\item
De ontwikkeling van drones introduceert een wapenwedloop. Hierdoor komen er meer wapens en wordt er meer geld besteed aan wapens.

%\item
%Door alleen aan te vallen met autonome drones verschuift de verantwoordelijkheid naar het beslissingsvermogen van de drones. Hierdoor wordt de verantwoordelijkheid voor gemaakte slachtoffers ontweken.

\item
De tegenstanders gaan eerder de tegenaanvallen uitvoeren in het land van de aanvallers, door middel van terroristische aanslagen. Anders raken ze de mensen in dat land niet.

\item
Gevechtsdrones zijn computers. Deze kunnen dus gehacked worden en misbruikt worden door de tegenstanders.

\end{itemize}

\subsection{Afwegingen}

\begin{itemize}
\item
Voorbeelden waar veel burgerslachtoffers gevallen zijn en ook voorbeelden waar juist helemaal geen burgerslachtoffers gevallen zijn.
\end{itemize}
