\section{Group Server}\label{Server}
Group Server was responsible for the communication between the server and the MySQL database, the communication between the ODROID and the server and for the dashboard page. For this project, the languages HTML, PHP, CSS, JavaScript and C were used.
\subsection{Dashboard}
The dashboard is the page where all of the data from the database can be viewed in graphs and tables. On the page is a checkbox menu to choose which data must be retrieved from the database. Because of this even users who don’t have any knowledge of SQL can retrieve data from the database. The dashboard page is adaptable every time the page is refreshed the whole menu will be rebuilt. So if there are new tables and/or new columns added to the database the menu will automatically add these to the checkbox menu. The checkbox menu also adds the capability to add requirements to the data that must be retrieved. The status of the chargers is showed on the page and can be changed manually by clicking on the status buttons.
\subsection{Server}
The communication between the server and the database has to be protected. There were three functions built to make sure that the database was protected against SQL injection attacks. These functions are insertQuery , updateQuery and getQuery. These functions check the input given by the user before sending the query to the database. If the false query doesn't get rejected by the functions then it wil be placed inside the database as a string. So it still doesn’t have any effect on the database
\subsection{Communication}
The communication between the server and the ODROID is done with the notation JSON, since it is one of the most flexible data structure notations. The ODROID wraps the data in a JSON array, adds some extra data (the current time and the type of data) in a JSON object, and adds the HTTP headers. A socket is opened, and the server's IP address is found. Then the ODROID tries to connect to the server. If the connection succeeds, the JSON object is send to the server. The server decodes the data, writes it into the database, or read things from the database, depending on the type of data. The server optionally sends data back, if it was read from the database. The ODROID reads it, decodes it, and gives it back to the main program.

If internet fails, the data cannot be send to the server. The ODROID then buffers the data. This buffer can handle 200 messages. If the 201st is inserted, the first is deleted.
That way, about 8 hours of data is preserved.
